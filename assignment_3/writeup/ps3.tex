%%%% CS 224N: Assignment #3 %%%%
%\documentclass[fleqn,10pt]{article}
%\usepackage{simplemargins}
%\setallmargins{1.0in}
\documentclass[10pt,reqno]{amsart}
\usepackage{amsmath}
\usepackage{amssymb}
\usepackage{amsthm}
\usepackage{bm}
\usepackage{enumitem}
\usepackage{graphicx}
\usepackage[paper=letterpaper,margin=0.6in]{geometry}
\usepackage[all]{xy}


%% Quick fix for wrapping text within a table column
\usepackage{array}
\newcolumntype{L}{>{\centering\arraybackslash}m{3cm}}


%% Define header
\begin{document}
\title{CS224n: Natural Language Processing with Deep Learning\\Assignment \#3}
\author{Anthony Ho}
\maketitle


%% Define shortcuts
\newcommand{\f}{\frac}
\newcommand{\pd}[1]{\frac{\partial}{\partial #1}}
\newcommand{\pdd}[2]{\frac{\partial #1}{\partial #2}}
\newcommand{\softmax}{\text{softmax}}


%% Set up numbering environment
\renewcommand{\labelenumi}{\arabic{enumi}.}
\begin{enumerate}[topsep=0pt,itemsep=3ex,partopsep=1ex,parsep=1ex]


%% Question 1
\item
  \begin{enumerate}[itemsep=2ex]
  %% Question 1(a)
  \item 
    \begin{enumerate}[itemsep=2ex]
      \item
        Example 1: ``Stanford is great.'' - where ``Stanford'' could refer to 
        Stanford University (organization) or a person with last name Stanford (person).

        Example 2: ``I am going to Stanford.'' - where ``Stanford could refer to 
        Stanford Unviersity (organization) or Stanford, California (location).
      \item Because the features apart from the word itself could provide context 
        that could remove ambiguity from the named entity.
      \item
        Feature 1: the feature ``in'' that immediately preceded a word could be helpful 
        for identifying the word as a location.

        Feature 2: an action verb that immediately succeeded a word could be helpful 
        for identifying the word as a named entity.
    \end{enumerate}
  %% Question 1(b)
  \item 
    \begin{enumerate}[itemsep=2ex]
      \item The dimensions are:
        \begin{align*}
          \bm{e}^{(t)} \in \mathbb{R}^{1 \times D(2w+1)} \\
          W \in \mathbb{R}^{D(2w+1) \times H} \\
          U \in \mathbb{R}^{H \times C} 
        \end{align*}
      \item The computational complexity of predicting labels for a sentence of length $T$ is
        $\mathcal{O}(T (D(2w+1)H + HC))$.
    \end{enumerate}
  %% Question 1(c)
  \item %Please see the coding portion of the assignment.
  %% Question 1(d)
  \item 
  \end{enumerate}


%% Question 2
\item
  \begin{enumerate}[itemsep=2ex]
  %% Question 2(a)
  \item 
  %% Question 2(b)
  \item 
  %% Question 2(c)
  \item %Please see the coding portion of the assignment.
  %% Question 2(d)
  \item 
  %% Question 2(e)
  \item %Please see the coding portion of the assignment.
  %% Question 2(f)
  \item %Please see the coding portion of the assignment.
  %% Question 2(g)
  \item 
  \end{enumerate}


%% Question 3
\item
  \begin{enumerate}[itemsep=2ex]
  %% Question 3(a)
  \item 
  %% Question 3(b)
  \item
  %% Question 3(c)
  \item %Please see the coding portion of the assignment.
  %% Question 3(d)
  \item %Please see the coding portion of the assignment.
  %% Question 3(e)
  \item 
  %% Question 3(f)
  \item %Please see the coding portion of the assignment.
  \end{enumerate}


\end{enumerate}
\end{document}
