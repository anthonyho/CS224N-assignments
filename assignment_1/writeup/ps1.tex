%%%% CS 224N: Assignment #1 %%%%
%\documentclass[fleqn,10pt]{article}
%\usepackage{simplemargins}
%\setallmargins{1.0in}
\documentclass[11pt,reqno]{amsart}
\usepackage{amsmath}
\usepackage{amssymb}
\usepackage{amsthm}
\usepackage{bm}
\usepackage{enumitem}
\usepackage[paper=letterpaper,margin=0.75in]{geometry}
\usepackage[all]{xy}

%% Define header
\begin{document}
\title{CS224n: Natural Language Processing with Deep Learning\\Assignment \#1}
\author{Anthony Ho}
\maketitle


%% Define shortcuts
\newcommand{\f}{\frac}
\newcommand{\pd}[1]{\frac{\partial}{\partial #1}}
\newcommand{\pdd}[2]{\frac{\partial #1}{\partial #2}}
\newcommand{\softmax}{\text{softmax}}


%% Set up numbering environment
\renewcommand{\labelenumi}{\arabic{enumi}.}
\begin{enumerate}[topsep=0pt,itemsep=2ex,partopsep=1ex,parsep=1ex]


%% Question 1
\item
  \begin{enumerate}
  %% Question 1(a)
  \item
    For any input vector $\bm{x}$ and any constant $c$,
    \begin{align}
      \softmax(\bm{x} + c)_i
      &= \f{e^{x_i + c}}{\sum_j e^{x_j + c}} \nonumber \\
      &= \f{e^c e^{x_i}}{\sum_j e^c e^{x_j}} \nonumber \\
      &= \f{e^c e^{x_i}}{e^c \sum_j e^{x_j}} \nonumber \\
      &= \f{e^{x_i}}{\sum_j e^{x_j}} \nonumber \\
      &= \softmax(\bm{x})_i \label{1a}
    \end{align}
    Since (\ref{1a}) is true for any arbitrary element $i$,
    we can conclude that:
    \begin{equation*}
      \softmax(\bm{x}) = \softmax(\bm{x} + c)
    \end{equation*}
    
  %% Question 1(b)
  \item Please see the coding portion of the assignment.
  \end{enumerate}


%% Question 2
\item
  \begin{enumerate}
  %% Question 2(a)
  \item 
    First, we can rearrange the definition of the sigmoid function to obtain:
    \begin{align*}
      e^{-x} = \f{1}{\sigma(x)} - 1
    \end{align*}
    Now we can derive the gradient of the sigmoid function w.r.t. $x$,
    assuming $x$ is a scalar. 
    \begin{align*}
      \pdd{\sigma(x)}{x}
      &= \pd{x} \f{1}{1 + e^{-x}} \\
      &= \f{-1}{(1 + e^{-x})^2} \left( - e^{-x} \right) \\
      &= \f{1}{(1 + e^{-x})^2} \left( e^{-x} \right) \\
      &= \left( \sigma(x) \right)^2 \left( \f{1}{\sigma(x)} - 1 \right) \\
      &= \sigma(x) \left( 1 - \sigma(x) \right)
    \end{align*}

  %% Question 2(b)
  \item
    First, let's consider the fact that $\bm{y}$ is the one-hot label vector, i.e.
    \begin{equation*}
      y_i = 
      \begin{cases}
        1, & \text{if}\ i=k \\
        0, & \text{otherwise}
      \end{cases}
    \end{equation*}
    where $k$ is the index of the true label. 
    
    Therefore, we can simplify the cross entropy function as:
    \begin{equation*}
      \text{CE}(\bm{y}, \bm{\hat{y}})
        = - \sum_i y_i \log(\hat{y}_i)
        = - \log(\hat{y}_k)
    \end{equation*}

    To derive the gradient w.r.t the inputs of a softmax function when cross entropy loss
    is used for evaluation, let's consider its individual elements:
    \begin{align}
      \pd{\theta_i} \text{CE}(\bm{y}, \bm{\hat{y}})
      &= \pd{\theta_i} \left[ - \log(\hat{y}_k) \right] \nonumber \\
      &= \pd{\theta_i} \left[ - \log(\f{e^{\theta_k}}{\sum_j e^{\theta_j}}) \right] \nonumber \\
      &= \pd{\theta_i} \left[ -\theta_k + \log \sum_j e^{\theta_j} \right] \nonumber \\
      &= - \pdd{\theta_k}{\theta_i} + \f{\sum_j e^{\theta_j} \pdd{\theta_j}{\theta_i}}{\sum_j e^{\theta_j}} \label{2b}
    \end{align}

    By noting that:
    \begin{equation*}
      \pdd{\theta_j}{\theta_i} = 
      \begin{cases}
        1, & \text{if}\ i=j \\
        0, & \text{otherwise}
      \end{cases}
    \end{equation*}

    We can simpify (\ref{2b}) as:
    \begin{align*}
      \pd{\theta_i} \text{CE}(\bm{y}, \bm{\hat{y}})
      &= - y_i + \f{e^{\theta_i}}{\sum_j e^{\theta_j}} \\
      &= - y_i + \hat{y}_i 
    \end{align*}

    Thus, the gradient w.r.t the inputs of a softmax function when cross entropy loss
    is used for evaluation is:
    \begin{equation*}
      \pd{\bm{\theta}} \text{CE}(\bm{y}, \bm{\hat{y}}) = \bm{\hat{y}} - \bm{y}
    \end{equation*}

  %% Question 2(c)
  \item
    Let's denote:
    \begin{align*}
      \bm{\theta_1} &= \bm{x W_1} + \bm{b_1} \\
      \bm{\theta_2} &= \bm{h W_2} + \bm{b_2}
    \end{align*}
    
    By applying chain rule, we can rewrite the gradient as:
    \begin{equation*}
      \pdd{J}{\bm{x}}
      = \pdd{\text{CE}(\bm{y}, \bm{\hat{y}})}{\bm{x}}
      = \pdd{\text{CE}(\bm{y}, \bm{\hat{y}})}{\bm{\theta_2}} \pdd{\bm{\theta_2}}{\bm{h}} \pdd{\bm{h}}{\bm{\theta_1}} \pdd{\bm{\theta_1}}{\bm{x}}
    \end{equation*}
    
    The first component is simply the result of part (b):
    \begin{equation*}
      \pdd{\text{CE}(\bm{y}, \bm{\hat{y}})}{\bm{\theta_2}} = \bm{\hat{y}} - \bm{y}
    \end{equation*}
    
    The second component is:
    \begin{equation*}
      \pdd{\bm{\theta_2}}{\bm{h}} = \pd{\bm{h}} \left( \bm{h W_2} + \bm{b_2} \right) = \bm{W_2}^\top
    \end{equation*}

    The third component uses the result of part (a):
    \begin{equation*}
      \pdd{\bm{h}}{\bm{\theta_1}} = \pdd{\sigma(\bm{\theta_1})}{\bm{\theta_1}} = \sigma'(\bm{\theta_1})
    \end{equation*}
    where $\sigma'(\bm{\theta_1})$ denotes a $H \times H$ diagonal matrix where 
    the diagonal elements are the derivatives of $\sigma(\bm{\theta_1})_{i}$ w.r.t. ${\theta_1}_i$, i.e.:
    \begin{equation*}
      \sigma'(\bm{\theta_1})_{ii}
      = \pdd{\sigma(\bm{\theta_1})_{i}}{{\theta_1}_i}
      = \pdd{\sigma({\theta_1}_i)}{{\theta_1}_i}
      = \sigma({\theta_1}_i) \left( 1 - \sigma({\theta_1}_i) \right)
    \end{equation*}

    The fourth component is similar to the second component:
    \begin{equation*}
      \pdd{\bm{\theta_1}}{\bm{x}} = \pd{\bm{x}} \left( \bm{x W_1} + \bm{b_1} \right) = \bm{W_1}^\top
    \end{equation*}

    Therefore, the the gradient with respect to the inputs $\bm{x}$ to an one-hidden-layer neural network is:
    \begin{align*}
      \pdd{J}{\bm{x}}
      &= \pdd{\text{CE}(\bm{y}, \bm{\hat{y}})}{\bm{\theta_2}} \pdd{\bm{\theta_2}}{\bm{h}} \pdd{\bm{h}}{\bm{\theta_1}} \pdd{\bm{\theta_1}}{\bm{x}} \\
      &= \left( \bm{\hat{y}} - \bm{y} \right) \bm{W_2}^\top \sigma'(\bm{\theta_1}) \bm{W_1}^\top \\
      &= \left( \bm{\hat{y}} - \bm{y} \right) \bm{W_2}^\top \sigma'(\bm{x W_1} + \bm{b_1}) \bm{W_1}^\top
    \end{align*}

  %% Question 2(d)
  \item
    The dimensions of the weights and biases are as follows:
    \vspace{1mm}
    \begin{center}
      \begin{tabular}{|c|c|}
        \hline
        Parameter & Dimension \\
        \hline
        $W_1$ & $D_x \times H$ \\
        $b_1$ & $1 \times H$ \\
        $W_2$ & $H \times D_y$ \\
        $b_2$ & $1 \times D_y$ \\
        \hline
      \end{tabular}
    \end{center}
    \vspace{1mm}
    Therefore, the number of parameters in this neural network is:
    \begin{equation*}
      \text{\# parameters} = D_x H + H + H D_y + D_y = (D_x + 1) H + D_y (H + 1)
    \end{equation*}

  %% Question 2(e)
  \item Please see the coding portion of the assignment.
  %% Question 2(f)
  \item Please see the coding portion of the assignment.
  %% Question 2(g)
  \item Please see the coding portion of the assignment.
  \end{enumerate}


%% Question 3
\item
  \begin{enumerate}
  %% Question 3(a)
  \item
    Let's denote:
    \begin{align*}
      \theta_w &= \bm{u}_w^\top \bm{v}_c \\
      \bm{\theta} &= \bm{U}^\top \bm{v}_c
    \end{align*}
    where $\theta_w$ is a scalar,
    $\bm{u}_w$ and $\bm{v}_c$ are column vectors of dimensions $N \times 1$,
    $\bm{\theta}$ is a column vector of dimension $V \times 1$, 
    and $\bm{U} = [\bm{u}_1, \bm{u}_2, \cdots, \bm{u}_V]$ is a matrix of dimension $N \times V$.

    The softmax predictions for every word can then be written as:
    \begin{equation*}
      \bm{\hat{y}} = \f{\exp(\bm{U}^\top \bm{v}_c)}{\sum_{w=1}^{V} \exp(\bm{u}_w^\top \bm{v}_c)}
      = \f{\exp(\bm{\theta})}{\sum_{w=1}^{V} \exp(\theta_w)}
    \end{equation*}
    where $\bm{\hat{y}}$ is a column vector of softmax predictions for every word of dimension $V \times 1$.

    By using chain rule and the result of 2(b), the gradient of the cross entropy cost w.r.t. $\bm{v}_c$ 
    can be derived as:
    \begin{align*}
      \pd{\bm{v}_c} J_\text{softmax--CE}
      &= \pdd{\bm{\theta}}{\bm{v}_c} \pdd{J}{\bm{\theta}} \\
      &= \pdd{\bm{U}^\top \bm{v}_c}{\bm{v}_c} \pdd{\text{CE}(\bm{y}, \bm{\hat{y}})}{\bm{\theta}} \\
      &= \bm{U} \left( \bm{\hat{y}} - \bm{y} \right)
    \end{align*}
    where $\bm{y}$ is a column vector of expected word of dimension $V \times 1$.
  %% Question 3(b)
  \item
    As in the previous part, we can apply chain rule and the result of 2(b):
    \begin{align}
      \pd{\bm{u}_k} J_\text{softmax--CE}
      &= \pdd{\bm{\theta}}{\bm{u}_k} \pdd{J}{\bm{\theta}} \nonumber \\
      &= \pdd{\bm{U}^\top \bm{v}_c}{\bm{u}_k} \pdd{\text{CE}(\bm{y}, \bm{\hat{y}})}{\bm{\theta}} \nonumber \\
      &= \pdd{\bm{U}^\top \bm{v}_c}{\bm{u}_k} \left( \bm{\hat{y}} - \bm{y} \right) \label{3b}
    \end{align}
    
    Rewriting matrix multiplication in (\ref{3b}) explicitly:
    \begin{align*}
      \pd{\bm{u}_k} J_\text{softmax--CE}
      &= \sum_j^V \left( \bm{\hat{y}} - \bm{y} \right)_j \left( \pdd{\bm{U}^\top \bm{v}_c}{\bm{u}_k} \right)_{\cdot j} \\
      &= \sum_j^V \left( \hat{y}_j - y_j \right) \left( \pdd{\bm{U}^\top \bm{v}_c}{\bm{u}_k} \right)_{\cdot j} 
    \end{align*}
    where $\left( \pdd{\bm{U}^\top \bm{v}_c}{\bm{u}_k} \right)_{\cdot j}$ is the $j$-th column of 
    $\pdd{\bm{U}^\top \bm{v}_c}{\bm{u}_k}$ which is a $N \times V$ matrix.
    It can be simplified to:
    \begin{equation*}
      \left( \pdd{\bm{U}^\top \bm{v}_c}{\bm{u}_k} \right)_{\cdot j} =
      \begin{cases}
        \bm{v}_c, & \text{if}\ j=k \\
        0, & \text{otherwise}
      \end{cases}
    \end{equation*}

    Therefore,the gradient can be simplified to:
    \begin{align*}
      \pd{\bm{u}_k} J_\text{softmax--CE} 
      &= \left( \hat{y}_k - y_k \right) \bm{v}_c
    \end{align*}

    Note that:
    \begin{equation*}
      \pd{\bm{u}_k} J_\text{softmax--CE} =
      \begin{cases}
        \left( \hat{y}_k - 1 \right) \bm{v}_c, & \text{if}\ k=o \\
        \hat{y}_k \bm{v}_c, & \text{otherwise}
      \end{cases}
    \end{equation*}    

  %% Question 3(c)
  \item
    Let's denote:
    \begin{align*}
      \theta_o &= \bm{u}_o^\top \bm{v}_c \\
      \theta_k &= - \bm{u}_k^\top \bm{v}_c 
    \end{align*}

    The gradient of the negative sampling loss w.r.t. $\bm{v}_c$ is:
    \begin{align*}
      \pd{\bm{v}_c} J_\text{neg−-sample}
      &= \pd{\bm{v}_c} \left[ -\log(\sigma(\bm{u}_o^\top \bm{v}_c)) - \sum_{k=1}^K \log(\sigma(-\bm{u}_k^\top \bm{v}_c)) \right] \\
      &= \pd{\bm{v}_c} \left[ -\log(\sigma(\theta_o)) - \sum_{k=1}^K \log(\sigma(\theta_k)) \right] \\
      &= - \f{1}{\sigma(\theta_o)} \pdd{\sigma(\theta_o)}{\theta_o} \pdd{\theta_o}{\bm{v}_c} 
         - \sum_{k=1}^K \f{1}{\sigma(\theta_k)} \pdd{\sigma(\theta_k)}{\theta_k} \pdd{\theta_k}{\bm{v}_c} \\
      &= - \f{1}{\sigma(\theta_o)} \sigma(\theta_o) (1 - \sigma(\theta_o)) \pdd{\theta_o}{\bm{v}_c} 
         - \sum_{k=1}^K \f{1}{\sigma(\theta_k)} \sigma(\theta_k) (1 - \sigma(\theta_k)) \pdd{\theta_k}{\bm{v}_c} \\
      &= (\sigma(\bm{u}_o^\top \bm{v}_c) - 1)  \pdd{\bm{u}_o^\top \bm{v}_c}{\bm{v}_c} 
         + \sum_{k=1}^K (\sigma(-\bm{u}_k^\top \bm{v}_c) - 1) \pdd{(-\bm{u}_k^\top \bm{v}_c)}{\bm{v}_c} \\
      &= (\sigma(\bm{u}_o^\top \bm{v}_c) - 1) \bm{u}_o - \sum_{k=1}^K (\sigma(-\bm{u}_k^\top \bm{v}_c) - 1) \bm{u}_k
    \end{align*}

    Similarly, the gradient of the negative sampling loss w.r.t. $\bm{u}_o$ is:
    \begin{align*}
      \pd{\bm{u}_o} J_\text{neg−-sample}
      &= (\sigma(\bm{u}_o^\top \bm{v}_c) - 1)  \pdd{\bm{u}_o^\top \bm{v}_c}{\bm{u}_o} 
         + \sum_{k=1}^K (\sigma(-\bm{u}_k^\top \bm{v}_c) - 1) \pdd{(-\bm{u}_k^\top \bm{v}_c)}{\bm{u}_o} \\
      &= (\sigma(\bm{u}_o^\top \bm{v}_c) - 1) \bm{v}_c
    \end{align*}

    And the gradient of the negative sampling loss w.r.t. $\bm{u}_k$ is
    (changing summation indices from $k$ to $j$ to avoid confusion):
    \begin{align*}
      \pd{\bm{u}_k} J_\text{neg−-sample}
      &= (\sigma(\bm{u}_o^\top \bm{v}_c) - 1)  \pdd{\bm{u}_o^\top \bm{v}_c}{\bm{u}_k} 
         + \sum_{j=1}^K (\sigma(-\bm{u}_j^\top \bm{v}_c) - 1) \pdd{(-\bm{u}_j^\top \bm{v}_c)}{\bm{u}_k} \\
      &= (1 - \sigma(-\bm{u}_k^\top \bm{v}_c)) \bm{v}_c
    \end{align*}

    This cost function is much more efficient to compute than the softmax-CE loss 
    because the computation of $\pdd{J}{\bm{v}_c}$ for softmax-CE loss scales as $V$
    while the computation of $\pdd{J}{\bm{v}_c}$ for negative sampling loss scales as $K$, 
    resulting in a speed-up ratio of $K/V$,
    which could make a huge difference if one has a big vocabulary.

  %% Question 3(d)
  \item
  %% Question 3(e)
  \item
  %% Question 3(f)
  \item
  %% Question 3(g)
  \item
  %% Question 3(h)
  \item
  \end{enumerate}


%% Question 4
\item
  \begin{enumerate}
  %% Question 4(a)
  \item
  %% Question 4(b)
  \item
  %% Question 4(c)
  \item
  %% Question 4(d)
  \item
  %% Question 4(e)
  \item
  %% Question 4(f)
  \item
  %% Question 4(g)
  \item
  \end{enumerate}

\end{enumerate}
\end{document}
